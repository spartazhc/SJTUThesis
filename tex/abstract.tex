% !TEX root = ../thesis.tex

\begin{abstract}
  直播与分享已经成为人们生活中的一部分。然而,当海量的媒体数据在网络上传输的时候,会受到网络带宽波动的影响,高带宽占用也会引发网络拥塞等问题;另一方面,在实时应用中,低延迟深刻影响了用户体验。在如今的直播行业中有着二八定律:20\%的热门视频流(简称头部流)占据了80\%的流量,对于占有头部流量的大主播,观众人数多,亟需更高编码效率的新视频编码标准应用以减少CDN(Content Delivery Network)流量的费用支出,然而,由于新一代视频编码标准的复杂度往往数倍于上一代编码标准,在编码效率得到提升的同时,也导致了更高的编码延时的问题,在实时应用中亟待解决。本课题旨在研究新的编码标准在直播系统中的低延时优化,以解决以上痛点。

  本文首先对视频编码标准进行调研,对比各视频编码标准后选择AV1作为视频编码标准,对流媒体传输协议、封装协议进行调研,综合考虑延迟性能与扩展性选择协议,以FFmpeg与SVT-AV1等开源软件搭建了AV1低延迟直播系统原型,并在此基础上对该原型直播系统进行了延迟的分析与优化。优化主要在两个维度进行,一方面从系统层面优化系统中不必要的延迟、降低由于传输协议导致的延迟等;另一方面深入SVT-AV1编码器内部,分析编码过程中编码器内部的延迟来源并加以优化。

  系统层面的优化主要包括对SVT-AV1的高启动延迟优化与FFmpeg中AV1的RTP封装实现,完成后,系统端到端延迟降低至2秒左右,此时延迟主要来源为SVT-AV1编码器的编码延迟,因此需要对SVT-AV1编码器内部进行优化。

  降低编码延迟的主要方法为提升并行效率与降低计算量。在SVT架构已经尽可能提升了并行效率的前提下,我们的优化方向主要是降低编码计算量。据此,本文提出了基于人眼视觉感知特性的超级块快速划分算法,首先从基于亮度适应性、对比度掩蔽效应等视觉特性的JND感知阈值模型简化得到块级JND感知阈值模型,并根据该块级JND模型计算与块划分相关的感知划分指标。以Twitch的游戏测试序列为优化对象,分析了游戏测试序列的感知划分指标统计特征,取合适的感知划分指标阈值,在编码过程中实现提前终止划分。实验表明,对1080P视频序列的一帧计算该块级JND模型的时间小于1.8ms,以损失少量编码效率的代价,所提算法能够在SVT-AV1编码器最快预设基础上减少15\%左右的编码时间。
  % TODO 加入最终系统的延迟结果

  % 中文摘要应该将学位论文的内容要点简短明了地表达出来,应该包含论文中的基本信息,
  % 体现科研工作的核心思想。摘要内容应涉及本项科研工作的目的和意义、研究方法、研究
  % 成果、结论及意义。注意突出学位论文中具有创新性的成果和新见解的部分。摘要中不宜
  % 使用公式、化学结构式、图表和非公知公用的符号和术语,不标注引用文献编号。硕士学
  % 位论文中文摘要字数为 500 字左右,博士学位论文中文摘要字数为 800 字左右。英文摘
  % 要内容应与中文摘要内容一致。

  % 摘要页的下方注明本文的关键词(4~6个)。
\end{abstract}

\begin{enabstract}
  Shanghai Jiao Tong University (SJTU) is a key university in China. SJTU was
  founded in 1896. It is one of the oldest universities in China. The University
  has nurtured large numbers of outstanding figures include JIANG Zemin, DING
  Guangen, QIAN Xuesen, Wu Wenjun, WANG An, etc.

  SJTU has beautiful campuses, Bao Zhaolong Library, Various laboratories. It
  has been actively involved in international academic exchange programs. It is
  the center of CERNet in east China region, through computer networks, SJTU has
  faster and closer connection with the world.
\end{enabstract}
