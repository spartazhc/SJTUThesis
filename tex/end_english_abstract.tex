% !TEX root = ../thesis.tex

\begin{bigabstract}

  Live broadcasting and sharing have become an important part of people's daily lives. However, when massive media data is transmitted on the network, it will be affected by network bandwidth fluctuations, and high bandwidth usage will also cause problems such as network congestion; on the other hand, in real-time applications, low latency profoundly affects the user experience. In today's live broadcast industry, there is a 80 / 20 rule in distribution of video workloads that 20\% of the most popular video streams make up 80\% of traffic. For these head channels, we urgently need some more efficient video coding standards to reduce the CDN (Content Delivery Network) cost. However, because the complexity of new generation video coding standards is often hundred times that of previous ones, while the coding efficiency is improved, it also causes higher coding latency. It needs to be solved in real-time applications. In this paper, will study the low-latency optimization of the new coding standard in the live broadcast system to solve the above pain points.

  The paper first introduced the core video encoding technology, including prediction, transform, quantization, rate-distortion optimization, entropy coding and so on. In practical applications, the above technologies are used in combination to achieve efficient video compression. Then the paper compared the most commonly used video encoding standards: H.264/AVC, VP9, H.265/HEVC, AV1, H.266/VVC. Considering the following reasons, AV1 was selected as the coding standard of the live broadcast system after comprehensive comparison:
  (1) AV1 is a royalty-free video coding standard, which has a huge advantage over the royalty and patent issues of HEVC;
  (2) AV1 is the latest generation of encoding standard, with high encoding efficiency;
  (3) AV1 is designed to be used for video transmission on the Internet, suitable for video live application scenarios.
  AV1 is an open, royalty-free video coding format designed for video transmissions over the Internet. It uses a classic hybrid video encoding framework, including many advanced encoding tools including more diverse blocking schemes, more intra prediction directions, chroma from luminance, color palette mode and intra blocks copy for screen content, more reference frames pool, more flexible motion vector estimation and other inter prediction tools. For loop filtering and post-processing, tools such as CDEF, frame super-resolution, and film grain synthesis are used. These advanced coding tools enable AV1 to achieve higher coding efficiency.

  After investigating streaming transmission protocols and containers, and considering latency and scalability, SRT is selected as the upstream protocol and RTSP is the downstream protocol. The construction of the prototype live broadcast system is the basis for subsequent optimization. In order to achieve the RTSP protocol transmission of AV1, the RTP mux and demux logic for the AV1 code stream are implemented in FFmpeg. Based on FFmpeg and SVT-AV1 and other open source software, the AV1 low-latency live broadcast system prototype was built, and on this basis, the prototype live broadcast system was analyzed and optimized for latency. The optimization is mainly carried out in two dimensions. On the one hand, the paper optimizes the unnecessary latency in the system from the system level and reduces the latency caused by the transmission protocol. On the other hand, by going deep into the SVT-AV1 encoder, the internal latency of the encoder is analyzed and optimized.

  The system-level optimization is mainly to optimize the high startup latency of SVT-AV1. Two optimization methods are used: optimizing the call of malloc function to improve the efficiency of memory allocation; pre-launching the SVT-AV1 encoder in FFmpeg to eliminate the startup latency. In addition, by adjusting the configuration of x264 encoder and ffplay and other components also reduce the system latency. After the system-level optimization is completed, the end-to-end latency of the prototype system is reduced to about 2 seconds. At this time, the main source of the latency is the encode latency caused by the SVT-AV1 encoder, so it is necessary to optimize the SVT-AV1 encoder to get lower end-to-end latency.

  In the next part of the paper, the SVT architecture is introduced. Scalable Video Technology (SVT) is a software-based video coding technology that allows encoders to achieve the best possible tradeoffs between performance, latency, and visual quality. The SVT architecture allows for the encoder core to be split into independently operating threads, each thread processing a different segment of the input picture that run in parallel on different processor cores, without introducing any loss in fidelity. Parallelism in the encoder could be achieved at multiple levels: process level, picture level and segment level. In order to analyze the distribution of encoding latency of SVT architecture, the latency-profile tool for SVT is designed. Using the encoding latency-profile tool to profile the SVT-AV1 encoders. According to the analysis results, an optimization is done to eliminate the 1 frame input latency caused by the EOS mechanism in the RESOURCE process. By reducing the sliding window of scene changing detection in the PD process, large amount of PD process schedule latency is optimized. Tile encoding is enabled, the paper analyzed how the different tile split selections influences the encoding efficient in BD-Rate and the encoding latency. The analysis provides guidance for the subsequent selection of the optimal encoding efficiency - encodeing latency trade-off tile partitioning scheme.

  The method to reduce the codec latency is to increase the parallel efficiency and reduce the amount of calculation. Under the premise that the SVT architecture has improved parallel efficiency as much as possible, our optimization direction is to reduce the amount of coding calculation. This paper proposes a fast partition algorithm based on human visual perception characteristics. First, the paper studied the pixel-level JND perception threshold model based on visual characteristics such as luminance adaptability and contrast masking effect. In order to improve the calculation speed of the JND perception threshold model, the pixel-level model is simplified to obtain the block-level JND perception threshold model, and SIMD vector optimization is applied to further accelerate the calculation speed. Experiments show that the time to calculate the block-level JND model for a frame of 1080P video is less than 1.8ms.

  The paper then proposes a perceptual partitioning indicators (PI) related to partitioning. Taking the game test sequence of Twitch as the optimization object. The video game sequence belongs to the screen content, in which artificial textures are used. These textures are more monotonous than natural scene video sequences, and the similarity between textures is high. Monotonous textures often have high continuity. Therefore, in the super block partition, the game sequence has more large blocks with a lower partition depth than the natural scene sequence. Using the JND-based perception method can be effectively applied to the super block partition fast algorithm of the game sequence. The statistical characteristics of PI of the game test sequence are analyzed, it is found that the perceptual partition index PI of the non-partition mode is relatively small, and has a certain distinction from the perceptual partition index PI of the partition mode. Therefore, the perceptual partition index PI can be used to guide the early termination of block division. And the appropriate perceptual threshold (PT) of the PI is selected. Partition can be  terminated early in the coding process when the block PI is lower then PI. Experiments show that at the cost of 2\% of coding efficiency, the proposed algorithm can reduce 15\% encoding time on the basis of SVT-AV1's fastest preset. Test is on the both GoP structure of All Intra and Low Delay P.
\end{bigabstract}
