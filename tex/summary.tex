% !TEX root = ../thesis.tex

\begin{summary}

  \section{总结}
  本文首先对视频编码进行了简要介绍,对比分析了常用的视频编码标准,特别介绍了AV1视频编码标准。介绍了流媒体传输协议,包括基于HTTP的自适应流媒体协议如DASH / HLS,与有状态的流媒体协议如RTMP、RTSP、SRT等。结合AV1对媒体封装协议的支持性与各流媒体协议的延迟性能、扩展性,综合考虑选择了SRT作为上行协议,RTSP作为下行协议。基于FFmpeg搭建了AV1低延迟直播系统原型。

  对于所搭建的AV1低延迟直播系统原型,从系统层面与编码器层面对延迟进行了分析与优化。系统层面的优化包括解决SVT-AV1编码器高启动延迟的预启动优化、FFmpeg中AV1的RTP封装逻辑实现等。为SVT架构提出了延迟profile工具,据此对SVT-AV1编码器的延迟来源进行了仔细的分析,并根据延迟分析结果进行优化,包括:
  \begin{enumerate}
  	\item 对RESOURCE过程的EOS优化,解决了EOS机制导致的1帧输入延迟;
  	\item 消除了在PD过程中因为滑动窗口机制产生的高PD\_sche延迟;
  	\item 并分析了不同Tile划分对编码效率和编码延迟的影响。
  \end{enumerate}

  本文针对人类视觉特性提出了基于JND的快速划分算法,首先介绍了人类视觉特性,然后使用改进的块级JND感知阈值模型计算得到感知划分因子以指导块划分。面向Twitch的游戏序列统计分析得到感知划分因子阈值。基于感知变化因子在SVT-AV1的最快编码预设模式基础上建立了块划分快速算法,实现超级块划分的提前终止,降低大量的编码复杂度。实验结果表明,在SVT-AV1的最快编码预设的基础上,在损失2\%左右编码性能的前提下,本章所提出的超级块快速划分算法在All Intra和Low Delay P两种GoP结构上均可以减少15\%左右的编码时间。

	测试结果表明,应用本文所提的多种优化方式后,SVT-AV1以3\%左右的编码性能代价可换取超过50\%的编码延迟降低。经过优化的直播系统原型局域网下总延迟从将近10s降低至1s。

	\section{展望}
	本文搭建了基于AV1编码标准的低延迟直播系统原型,并从系统层面以及SVT-AV1编码器内部对延迟进行了优化。在本文工作的基础上,还有许多提升空间,以下从系统层面与编码器层面加以阐述:

	对于直播系统层面,所搭建的直播系统原型本身十分简陋,因为本文重点在于对SVT-AV1编码器的优化,实现直播系统的目的是对AV1低延迟直播的探索。在实际落地时,应该考虑业务特点,改用DASH/HLS做主要流量的分发,并部署边缘节点用做对部分用户的低延迟直播支持。

	在对SVT-AV1编码器的延迟优化上,本文探索了编码器的低延迟配置,对调度产生的延迟进行了优化,并提出了基于JND模型的超级块划分快速算法。在后续的工作中,可以在以下方面继续探索:
	\begin{enumerate}
		\item 实现对AV1编码的SVC可伸缩视频编码(Scalable Video Coding, SVC)支持,使用SVC加上AV1的切换帧(Switch Frame),在自适应比特率流传输中可以获得较高增益;
		\item 调整SVT架构对不同处理器情况下以及不同预设情况下的线程数分配,提高SVT架构的在不同处理器上的伸缩性;
		\item 优化SVT-AV1对不同预设的编码工具与使用算法的配置;
		\item 优化SVT-AV1的码率控制,SVT-AV1的多线程框架下对码率控制提出了更高的要求,需要更好的算法实现码率控制;
		\item 使用决策树等技术根据序列特征调整感知划分阈值,提高基于JND的超级块快速划分算法准确性。
	\end{enumerate}
\end{summary}
