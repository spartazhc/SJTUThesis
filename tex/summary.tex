% !TEX root = ../thesis.tex

\begin{summary}

  本文首先研究了流媒体传输协议,包括基于HTTP的自适应流媒体协议如DASH / HLS,与有状态的流媒体协议如RTMP、RTSP、SRT等。结合AV1对媒体封装协议的支持性与各流媒体协议的延迟性能、扩展性,综合考虑选择了SRT作为上行协议,RTSP作为下行协议。基于FFmpeg搭建了AV1低延迟直播系统原型。

  对于所搭建的AV1低延迟直播系统原型,从系统层面与编码器层面对其进行了充分优化。系统层面的优化包括解决SVT-AV1编码器高启动延迟的预启动优化、FFmpeg中AV1的RTP封装逻辑实现灯光。编码器层面的优化包括对RESOURCE过程中EOS机制导致的1帧输入延迟的优化,分tile编码时tile切分的优化、以及ENCDEC过程中基于JND的快速划分算法优化。

  本文针对人类视觉特性提出了基于JND的快速划分算法,首先介绍了人类视觉特性,然后使用基于亮度适应性、对比度掩蔽效应和视觉模式复杂度等视觉特性的多维 JND 感知阈值模型可以计算出与超级块划分具有相关性的感知划分因子。在twitch的游戏序列中搜索统计特性一般的感知划分影子阈值。基于感知变化因子在SVT-AV1的最快编码预设模式基础上建立了块划分快速算法,实现超级块划分的提前终止,以少量编码性能为代价降低了大量的编码复杂度。实验结果表明,视觉感知特性的超级块快速算法能够All Intra和Low Delay P两种GOP结构下减少接近15\%的编码时间,进一步调整md-stage后可以在All Intra下减少30\%的编码时间。




\end{summary}
