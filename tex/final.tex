\chapter{系统总体评测}
为了评估对SVT-AV1编码器以及直播系统进行的优化,在本章将对前述章节所提的优化方案综合运用,测试在延迟、编码性能上的结果。

\section{对SVT-AV1编码器的优化总测试}
	对SVT-AV1编码器面向编码延迟的优化包括以下几个方面:
	\begin{enumerate}[label=\arabic*)]
		\item 对RESOURCE过程的EOS优化;
		\item 对PD过程的滑动窗口优化;
		\item 分Tile编码优化;
		\item 对ENCDEC过程使用基于JND的超级块快速划分算法优化。
	\end{enumerate}

	对SVT-AV1编码器的优化总测试在转码服务器上完成,测试环境与前述表\ref{tab:os-8280}与表\ref{tab:svt-lat}一致。以SVT-AV1普通低延迟配置:\texttt{--preset 8 --lookahead 0 --pred-struct 0}作为参考。

	测试结果如表\ref{tab:svt-final-test}所示,在所有十个序列上,在延迟方面有超过50\%的降低,在编码性能上存在3\%左右的损失,BD-Rate-VMAF相比BD-Rate-PSNR损失更低,说明优化算法一定程度上考虑了人眼视觉特性。
	\begin{table}[!hpt]
		\renewcommand{\arraystretch}{0.9}
		\caption{应用多种优化方式对SVT-AV1编码器的延迟(ms)优化结果}
		\label{tab:svt-final-test}
		\centering
		\begin{tabular}{|c|c|c|c|c|c|c|} \hline
			序列    & QP & 优化前延迟 &优化后延迟 & $\Delta$latency& \makecell{BD-Rate\\(PSNR)} & \makecell{BD-Rate\\(VMAF)}\\ \hline

			\multirow{4}*{CSGO} & 23 & 193 & 103 & -46.63 \% & \multirow{4}*{+5.78\%} & \multirow{4}*{+5.94\%} \\ \cline{2-5}
			& 31 & 170 & 88 & -48.24 \%&  & \\ \cline{2-5}
			& 39 & 151 & 75 & -50.33 \%&  & \\ \cline{2-5}
			& 47 & 138 & 66 & -52.17 \%&  & \\ \hline
			\multirow{4}*{DOTA2} & 23 & 369 & 167 & -54.74 \% & \multirow{4}*{+3.78\%} & \multirow{4}*{+3.14\%} \\ \cline{2-5}
			& 31 & 279 & 136 & -51.25 \%&  & \\ \cline{2-5}
			& 39 & 202 & 119 & -41.09 \%&  & \\ \cline{2-5}
			& 47 & 177 & 93 & -47.46 \%&  & \\ \hline
			\multirow{4}*{EuroTruckSim2} & 23 & 544 & 206 & -62.13 \% & \multirow{4}*{+2.42\%} & \multirow{4}*{+2.98\%} \\ \cline{2-5}
			& 31 & 454 & 168 & -63.00 \%&  & \\ \cline{2-5}
			& 39 & 308 & 165 & -46.43 \%&  & \\ \cline{2-5}
			& 47 & 231 & 126 & -45.45 \%&  & \\ \hline
			\multirow{4}*{FALLOUT4} & 23 & 546 & 244 & -55.31 \% & \multirow{4}*{+2.19\%} & \multirow{4}*{+3.06\%} \\ \cline{2-5}
			& 31 & 412 & 200 & -51.46 \%&  & \\ \cline{2-5}
			& 39 & 316 & 143 & -54.75 \%&  & \\ \cline{2-5}
			& 47 & 237 & 111 & -53.16 \%&  & \\ \hline
			\multirow{4}*{GTAV} & 23 & 399 & 201 & -49.62 \% & \multirow{4}*{+3.45\%} & \multirow{4}*{+3.54\%} \\ \cline{2-5}
			& 31 & 264 & 145 & -45.08 \%&  & \\ \cline{2-5}
			& 39 & 208 & 112 & -46.15 \%&  & \\ \cline{2-5}
			& 47 & 151 & 88 & -41.72 \%&  & \\ \hline
			\multirow{4}*{HEARTHSTONE} & 23 & 220 & 97 & -55.91 \% & \multirow{4}*{+2.83\%} & \multirow{4}*{+3.57\%} \\ \cline{2-5}
			& 31 & 197 & 92 & -53.30 \%&  & \\ \cline{2-5}
			& 39 & 157 & 70 & -55.41 \%&  & \\ \cline{2-5}
			& 47 & 150 & 75 & -50.00 \%&  & \\ \hline
			\multirow{4}*{MINECRAFT} & 31 & 704 & 287 & -59.23 \% & \multirow{4}*{+1.80\%} & \multirow{4}*{+0.38\%} \\ \cline{2-5}
			& 39 & 540 & 220 & -59.26 \%&  & \\ \cline{2-5}
			& 47 & 388 & 180 & -53.61 \%&  & \\ \cline{2-5}
			& 55 & 275 & 147 & -46.55 \%&  & \\ \hline
			\multirow{4}*{RUST} & 23 & 564 & 224 & -60.28 \% & \multirow{4}*{+4.11\%} & \multirow{4}*{+5.32\%} \\ \cline{2-5}
			& 31 & 381 & 163 & -57.22 \%&  & \\ \cline{2-5}
			& 39 & 282 & 128 & -54.61 \%&  & \\ \cline{2-5}
			& 47 & 198 & 105 & -46.97 \%&  & \\ \hline
			\multirow{4}*{STARCRAFT} & 23 & 438 & 212 & -51.60 \% & \multirow{4}*{+3.14\%} & \multirow{4}*{-2.78\%} \\ \cline{2-5}
			& 31 & 354 & 180 & -49.15 \%&  & \\ \cline{2-5}
			& 39 & 274 & 156 & -43.07 \%&  & \\ \cline{2-5}
			& 47 & 210 & 109 & -48.10 \%&  & \\ \hline
			\multirow{4}*{WITCHER3} & 23 & 616 & 256 & -58.44 \% & \multirow{4}*{+4.88\%} & \multirow{4}*{+4.09\%} \\ \cline{2-5}
			& 31 & 476 & 202 & -57.56 \%&  & \\ \cline{2-5}
			& 39 & 289 & 156 & -46.02 \%&  & \\ \cline{2-5}
			& 47 & 232 & 117 & -49.57 \%&  & \\ \hline
			\multicolumn{2}{|c|}{平均值} & 403 & 183 & -53.27\% & +3.44\% & +2.92\%

			\\\hline
		\end{tabular}
	\end{table}

\section{系统延迟总体测试}
	在局域网内对所搭建的低延迟直播系统进行测试,测试方法为在主播端编码前与客户端解码后打时间戳,对相应帧对比计算。与端到端的延迟相比,所测得的系统延迟缺少图像采集与解码后播放的延迟。测试结果如表\ref{tab:sys-lat}所示,其中,转码服务器\texttt{copy}表示不进行转码,x265的配置为\texttt{-preset ultrafast -tune zerolatency}。%编码延迟极低但画面质量较差。
  \begin{table}[H]
		\caption{系统总体延迟测试}
		\label{tab:sys-lat}
		\centering
		\begin{tabular}{cccc}
			\toprule
			 转码编码器  & 编码器预启动 & 编码器调优 & 系统延迟(ms) \\ \midrule
			 copy   &   /    &   /   &   610    \\
			 x265   &   /    &   /   &   1210   \\
			SVT-AV1 &   否    &   否   &   9560   \\
			SVT-AV1 &   是    &   否   &   3650   \\
			SVT-AV1 &   否    &   是   &   8160   \\
			SVT-AV1 &   是    &   是   &   1300   \\ \bottomrule
		\end{tabular}
	\end{table}

	测试结果表明,前述对直播系统的延迟优化以及对SVT-AV1编码器的多方面优化能够有效降低系统延迟。SVT-AV1编码器的预启动对延迟有较大帮助,相比未预启动可减少6秒多的延迟。转码编码器使用copy的延迟情况可以表现由于主播端编码上传的延迟以及传输协议产生延迟。引入x265转码作为参考,表明由于转码引起的一些缓冲区延迟对系统延迟的影响。最终的优化结果在使用SVT-AV1做转码时可以达到1300ms的系统延迟
\section{本章小结}
在这一章中,对所构建的AV1低延迟直播系统进行了延迟与编码效率上的评测。测试结果表明,对SVT-AV1的优化对比其普通的低延迟配置,平均有超过50\%的延迟降低,编码性能下降在3\%左右,其中BD-Rate-VMAF相比BD-Rate-PSNR损失更低。在整个直播系统中,应用系统优化与优化后的SVT-AV1编码器,在局域网内的系统延迟降低至1300ms。